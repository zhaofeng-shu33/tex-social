\documentclass[12pt]{ctexart}
\def\version{4.6}
\newenvironment{smaller}{\zihao{5}}{\zihao{-4}}
% set the left/right margin such that the main title can be written within one line
\usepackage[left=30mm]{geometry}
\usepackage{enumitem}
\AddEnumerateCounter{\chinese}{\chinese}{}
\usepackage{fancyhdr}
\usepackage{graphicx}
\usepackage{longtable}
\fancypagestyle{runningpage}
{
  \fancyhead{}
  \fancyhead[C]{清华大学深圳研究生院紫荆志愿者团}
  \fancyfoot{}
  \fancyfoot[C]{第 \thepage 页}
}
% not works?
\ctexset {
 appendixname = {附录}
}
\def\CurriculumScheduleWidth{1.6cm}
\begin{document}
% first page is cover
\begin{titlepage}
\begin{center}
    \vspace{-0.5in}
    \textmd{\textbf{\huge{2018暑假清华大学深圳研究生院紫荆志愿者团爱心支教}}}\\
    \normalsize\vspace{0.1in}\Large{2018年春季学期}\\
    \vspace{1in}
     \textbf{\huge{策}}\\
    \vspace{1in}
     \textbf{\huge{划}}\\
    \vspace{1in}
     \textbf{\huge{书}}\\
    \vspace{1in}
策划人:赵丰
\vspace{0.2in}
\begin{smaller}

版本:\texttt{\version}

\today
\end{smaller}
\end{center}
\end{titlepage}
\thispagestyle{empty}
\pagebreak
\pagestyle{runningpage}

\section{项目背景}
海南省“美在心灵”大学生支教志愿者协会在寒暑假与高校团队对接、共同组织开展在海南的支教活动由来以久。

大学生支教活动作为对国家偏远地区教育扶贫的一项有益补充,主要以本科生为主。以国家“西部计划”为号召,全国各大高校组织已保研的本科生和部分研究生每年会开展为期一年的长期支教活动。与长期支教相辅相成的是每年寒暑假开展的短期支教活动。短期异地支教活动
%由于要在孩子在放假的时候来学校上课,在偏远且交通不便的地方孩子的安全是个问题,且由于支教时间短,在传授文化课方面支教志愿者可能做得并不如当地教师好
目前在社会上受到一定的非议,部分原因是个别团队不够专业、不能因地制宜与当地志愿团队、当地家长、当地学校密切配合所致。

清华大学致力于培养科研与德育全面发展的高素质人才,深圳研究生院紫荆志愿者团作为院团委的直属组织,始终坚持先进青年助人为乐、奉献社会的精神,是一支优秀的志愿团队。志愿者的成员不仅积极参与院内外的各类志愿活动,还负责与相关部门对接,招募志愿者,组织、协调活动。本次实践立项为志愿团成员商议后提出的想法,希望这次活动能探索高校研究生团队短期异地支教的思路,响应党的“十九大”基层扶贫的号召,以“教育扶贫”为主题,深入当地开展支教活动。

我们希望在本次支教活动中可以进一步深入调研当地基础教育的建设情况,形成有价值的调研报告,在为当地基层教育事业作出切实贡献的同时,也能够为当地基础教育的长远发展提供参考。
\begin{table}[!ht]
\centering
\begin{tabular}{|p{2cm}|p{2cm}|p{2cm}|p{3cm}|}
\hline
名称 & 负责人 & 来源 & 备注 \\
\hline
儿童机器人套装 & 赵丰 & 深圳市搭搭乐乐文化传播有限公司 & 11套 \\
\hline
素质拓展道具 & 黄少平 & 清华深研院心理辅导中心 & 6件 \\
\hline
DOBOT魔术师 & 黄少平 & 越疆科技 & 1件 \\
\hline
娃哈哈饮料 & 张炎武 & 打稔村购买 & 1箱,家访扶贫 \\
\hline
文具奖品 & 赵丰、张炎武、周俞君 & 自带、购买 & 课堂奖励 \\
\hline
Canon 相机 & 王鲜俐 & 自带 & 活动拍照 \\
\hline
\end{tabular}
\caption{物资一览表}
\end{table}
本次支教借用了深圳市搭搭乐乐文化传播有限公司11套儿童机器人套装,支教结束后4件套装运回,其余捐献给支教地;另外本次支教也借用了本院心理辅导中心用于素质拓展活动的道具,支教结束后道具全部寄回。结合以上教具,我们希望能开展更加丰富多样的支教活动,在给当地小朋友带去快乐的同时,能开拓他们的视野,丰富他们的精神世界。
\section{项目目标}
\begin{itemize}
\item 本院:扩大志愿团志愿服务的范围,进一步宣传院团委等。
\item 参与者:体验短期支教生活,培养吃苦耐劳、乐于奉献的精神,形成对科研生活的有益补充。
\item 支教地:提升支教地重视儿童教育的氛围,为当地基础教育的发展作出切实贡献。
\item 支教学生:打开孩子的视野,促进孩子的心灵成长。
\end{itemize}
\section{项目简介}
\begin{itemize}
\item 主办方:清华大学深圳研究生院紫荆志愿者团、海南省儋州市海头镇红洋村村委会
\item 协办方:美在心灵大学生支教志愿者协会
\item 地点:海南省儋州市海头镇红洋村
\item 时间:2018年7月30日到2018年8月5日,在支教地一周
\item 参与人数: 清华大学深圳研究生院学生6人,琼籍大学生2人
\begin{table}[!ht]
\centering
\begin{tabular}{|c|c|c|c|c|}
\hline
姓名 & 性别 & 年级 & 系别 & 联系方式\\
\hline
赵丰 & 男 & 研一 & 电子系 &  ***\\
\hline
张晓冰  &	男	& 研一 &  生物系 &  ***\\
\hline
张炎武 & 男 & 研一 & 医院管理 & ***\\
\hline
白杨 & 女 & 研一 & 电子系 & *** \\
\hline
王鲜俐 & 女 & 研一 & 材料系 & ***\\
\hline
黄少平 & 男 & 研一 & 自动化系 & ***\\
\hline
周俞君(琼) & 女 & 大一 & 学前教育 &  ***\\
\hline
周让强(琼) & 女 & 大二 & 医药营销 &  ***\\
\hline
\end{tabular}
\caption{参与支教志愿者名单}
\end{table}
\end{itemize}
\section{项目准备}
\begin{enumerate}
\item 志愿者招募
\item 筹备物资
\item 志愿者准备教案
\item 志愿者行前碰面会两次
\item 志愿者部分统一购买车票、出行保险
\end{enumerate}
\section{项目流程}

\begin{table}[!ht]
\centering
\begin{tabular}{|c|c|}
\hline
时间 & 安排 \\
\hline
7月28日下午 &  从深圳出发,经广州抵达海口 \\
\hline
7月29日下午 & 从海口东抵达白马井站,再转至红洋村 \\
\hline
7月30日上午 & 开营仪式\\
\hline
7月30日下午至8月5日上午 & 日常支教活动,穿插家访、调研 \\
\hline
8月5日下午 & 结营仪式 \\
\hline
8月5日晚上 & 志愿者返校或去其他地方 \\
\hline
\end{tabular}
\caption{行程安排表}\label{route}
\end{table}

\begin{longtable}{|p{\CurriculumScheduleWidth}|p{\CurriculumScheduleWidth}|p{\CurriculumScheduleWidth}|p{\CurriculumScheduleWidth}|p{\CurriculumScheduleWidth}|p{\CurriculumScheduleWidth}|p{\CurriculumScheduleWidth}|p{\CurriculumScheduleWidth}|}
\hline
日期 & 7月30日 & 7月31日 & 8月1日 & 8月2日 & 8月3日 & 8月4日 & 8月5日 \\
\hline
& 星期一 & 星期二 & 星期三 & 星期四 & 星期五 & 星期六 & 星期天 \\
\hline
8:00-8:40  &  & 素质拓展(黄少平,大班)& 刮画(周俞君,大班)  & 孟母三迁(张炎武,大班)三字经(张晓冰,小班) & 材料科学(王鲜俐,大班)动物(张晓冰,小班)& 神话故事(白杨、大班)人文景观(周让强、小班) & 机械臂激光刻字(黄少平,大班)  \\
\hline
9:00-9:40 & &  素质拓展(黄少平,大班)& 中国地理(张炎武,大班) & 搭小车(赵丰,大班)刮画(周俞君,小班)  & 素质拓展(黄少平,大班)唱歌(周让强,小班) & 机械臂(黄少平,大班)古诗两首(张晓冰,小班) & 邦宝积木(赵丰,大班) \\
\hline
10:00-10:40 &  &  素质拓展(黄少平,大班) & &  搭小车(赵丰,大班)唱歌(周让强,小班)  &  素质拓展(黄少平,大班)折青蛙(张晓冰,小班)& 机械臂(黄少平,大班)千纸鹤(赵丰,小班)&    邦宝积木(赵丰,大班) \\
\hline
3:00-3:40  & 简单机械模型(赵丰,大班) &  &  &  &  &  & \\
\hline
4:00-4:40  & 遥控车模型(赵丰,大班) & &  &  &  & &  \\
\hline
\caption{旧洋村:课程表}\label{curriculum_schedule}
\end{longtable}

\begin{longtable}{|p{\CurriculumScheduleWidth}|p{\CurriculumScheduleWidth}|p{\CurriculumScheduleWidth}|p{\CurriculumScheduleWidth}|p{\CurriculumScheduleWidth}|p{\CurriculumScheduleWidth}|p{\CurriculumScheduleWidth}|p{\CurriculumScheduleWidth}|}
\hline
日期 & 7月30日 & 7月31日 & 8月1日 & 8月2日 & 8月3日 & 8月4日 & 8月5日 \\
\hline
& 星期一 & 星期二 & 星期三 & 星期四 & 星期五 & 星期六 & 星期天 \\
\hline
9:00-9:40 & 开营仪式 &   & &  & &  &   \\
\hline
3:00-3:40  &  & 折纸(王鲜俐,小班) &  简单机械模型(黄少平,大小班) & 素质拓展(黄少平,大班)千纸鹤(赵丰,小班) & 搭小车(赵丰,大班),动物世界(张晓冰,小班) & 趣味英语(周俞君,大班)人文景观(周让强) &  结营仪式 \\
\hline
4:00-4:40  &  & & 遥控车模型(赵丰,大小班)  & 素质拓展(黄少平,大班)刮画(周俞君,大班) & 搭小车(赵丰,大班),唱歌(周让强,小班) & 机械臂(黄少平,大班)体育(张晓冰,小班) & \\
\hline
\caption{打稔村:课程表}\label{curriculum_schedule}
\end{longtable}

\section{项目后续阶段}
\begin{enumerate}
\item 整理资料,提交实践材料;完成往返火车票部分报销、出行保险报销;参与答辩等。
\item 重点联络两个孩子陈贵喜, 郭诏件:将两个孩子拉进QQ群“旧洋关爱组”,群内还有王鲜俐、黄少平、陈贵喜、[羊彩梅、陈帼帼](琼台旧洋支教队)。
\end{enumerate}
\section{联系方式}
清华大学深圳研究生院紫荆志愿者团副团长:赵丰

Tel: ***

Email: 616545598@qq.com
\begin{flushright}
清华大学深圳研究生院紫荆志愿者团\\
\the\year 年 \the\month 月 \the\day 日
\end{flushright}
\begin{appendix}
\section{关于院紫荆志愿者团}
清华大学深圳研究生院紫荆志愿者服务团是院团委直属的学生志愿服务组织,以“自我实践、服务他人、自我教育、推动社会”为宗旨,旨在组织、协调我院研究生志愿者活动,开展具有清华特色的丰富多彩的研究生志愿活动。
\begin{enumerate}[label = {\chinese*、}]
\item 常规志愿活动
\begin{enumerate}[label =(\arabic*)]
         \item  清华大学深圳研究生院新生集中办理义工证活动
         \item 与爱同行·关爱“星星的后裔“志愿活动
         \item  深圳高校“河未来、益起行”节水探源活动
         \item 院五星级志愿者评选活动
\end{enumerate}
\item 特色志愿活动
\begin{enumerate}[label =(\arabic*)]
        \item 大型赛会志愿服务活动
        \item 学院重点活动(如校庆、大学城半程马拉松、教工迎新年晚会等)志愿服务
        \item 面向留学生的志愿讲解员培养
\end{enumerate}
\item 突破性活动
\begin{enumerate}[label =(\arabic*)]
         \item 暑期支教
\end{enumerate}
\end{enumerate}
\section{关于美在心灵}
海南省“美在心灵”大学生支教志愿者协会(简称美在心灵)于2008年1月30日诞生于琼海市嘉积中学,2010年3月29日接受共青团海南省委业务主管,2011年7月11日在海南省民政厅注册成为合法民间公益机构,2017年12月6日被海南省民政厅授予“慈善组织”资格。美在心灵以“团结友爱,奉献社会”为宗旨,不以盈利为目的,广泛招募世界各地大学生志愿者到琼藏等乡村小学开展爱心支教活动,致力给孩子们一个快乐成长平台,给大学生一个社会实践机会,给社会一个务实且放心的爱心渠道,给政府一个温暖的辅助。目前已有美国、英国、俄罗斯、新加坡、韩国、日本等40多个国籍的大学生志愿者,以及清华、北大、港澳台等境内国内名校的大学生志愿者参加支教活动。十年以来,美在心灵安排志愿者1.9万人次,服务小学754所次(包括西藏两所小学:恩达小学和宾达小学),受益学生6.2万人次,总服务时长165万小时,总投入善款158万元。
\section{关于红洋村小学}
红洋村是一个行政村,下辖5个自然村,其中旧洋村和打稔村截止2018年暑假已经没有了小学,村委会在这两个自然村各设一个支教点由我们负责日常教学活动,每个支教点分大班和小班,大班是四至六年级,小班是一至三年级。同时参与支教的还有琼台师范学院的12名同学,她们支教重习惯养成,我们重开拓视野;经过协商我们上午8个人负责旧洋支教点,她们出4个人负责打稔支教点;下午我们8个人负责打稔支教点,她们出4个人负责旧洋村支教点的教学活动。由于两个支教点距离居住地“德立社区”较远,志愿者往返均为村书记等人亲自开车接送。
\section{关于“旧洋关爱”组}
在旧洋村支教点,紫荆团队支教7天,琼台师范学院4人支教10天,剩2人再支教10天。总体来说时间比较短暂,支教老师走后,很多孩子暴露出了一些问题。其中比较严重
的两个是陈贵喜和郭诏件。

因为小孩习惯用QQ交流,旧洋关爱“”组的实体是一个QQ群,是陈贵喜、郭诏件主动联系而自然形成的。

陈贵喜今年14岁,是旧洋大班上非常调皮的一个小男孩,染发,带手机来上课,上课东张西望,让志愿者老师大伤脑筋。我们进一步了解到,他没念完六年级就辍学了。当时他总是逃课去网吧,海头镇小学老师也管不了,让他父母领回家进行教育。后来他就随父母去了东莞,但他年纪太小,只能做些临时工。但由于没有户口,陈贵喜在东莞上不了学,只能每天游手好闲。今年暑假他随外出打工的父亲回到儋州,听说有个支教活动,非常乐意参加。但他在课上屡屡管不住自己,不过仍有小小的进步:在支教老师张炎武的劝说下,把流氓头给剃了。

我们走后,他打算跟他哥哥去东莞,或去深圳找亲戚,在赵丰和他爸爸的劝说下,又把他送回了海头镇的初中。在家教方面,陈贵喜经常好几天不回家,
他父母不给他钱的,他交到几个坏朋友
给他钱,但带他学坏。坏朋友指使他干活,不听话就打他。
陈贵喜受这类人的影响学坏了,我们走后还在琼台2位老师支教期间就发生了一件很不好的事。
陈贵喜用凳子打了
他堂弟陈贵宝,对他弟弟拳打脚踢。原因是他怀疑他弟弟偷了他手机。因为这件事,他爸爸很生气,但陈贵喜似乎没有表现出认错的态度。导致陈贵喜被他爸爸狠打一顿,他爸爸用木棍打他屁股,后来又打他的头,打出了血。

9月4日晚发生了陈贵喜向王鲜俐借钱买吃的这件事,经过赵丰向陈贵喜爸爸打电话确认,陈贵喜所述“爸爸去了姥姥家”系撒谎。从而最终没有借给他钱买吃的。陈贵喜一怒之下退了群。

黄少平从心理学的角度出发,认为“要让他有根本的改变,我认为只有两方面的改变才能让他变化。一是他的圈子,就是像赵丰说的去一个好学校,周围都是好同学,并且有严格的纪律约束他。二是,也是心理医生常用的方法。通过一些手段,配合家长来教育孩子,这各环节中最关键的是家长。如果家长不能配合的话,那么很难成功。”

陈贵喜父亲如果也去东莞打工的话,陈贵喜就成了典型的留守儿童。针对这一问题,赵丰咨询了海南留守儿童组儋州市联系人薛为榜老师,薛老师从老师的角度出发,认为
“他的老师必须用爱心来教育,给他精神上的鼓励,使他在快乐中生长。”

9月21日,陈贵喜用QQ给赵丰发了自己的定位,是在东莞火车站附近。台风过后,陈贵喜今年复学的愿景最终失败。赵丰、王鲜俐、张炎武三位志愿者利用中秋假期某一天的
的时间去东莞看望陈贵喜和其父母,进一步调研农民工子女孩子教育的艰难处境;下午带陈贵喜去东莞科技馆参观。
\section{项目财务预算}\label{scheduling}
\begin{enumerate}
\item 往返火车票
\item 住宿费,由当地政府帮忙解决。
\item 餐饮,由当地政府帮忙解决。
\item 宣传和教学纸制资料:必要的教具,横幅、教案打印、奖品。
\item 物资邮寄费用。
\item 志愿者出行保险。
\end{enumerate}
\section{志愿者保障和守则}
\subsection{保障}
\begin{enumerate}
\item 在志愿深圳平台上记24个工时。
\item 志愿者出行人身意外保险。
\item 支教地提供免费食宿。
\end{enumerate}
\subsection{守则}
\begin{enumerate}[label = {(\chinese*)}]
\item 具有完全民事行为能力的学生;
\item 认同紫荆志愿团理念和文化,尊重“美在心灵”的支教理念;
\item 服从支队长、村书记和相关爱心人士在支教期间的引导;
\item 有社会责任感和奉献精神,无不良嗜好;
\item 有较强的团结协作能力、语言表达能力和人际交往能力,尊重队友;
\item 有较强的心理素质,在艰苦的环境下能自我调节,保持积极乐观心态,肯吃苦,能坚持,不给支教项目及队友带来负面影响; 
\item 有爱心,热心帮助当地的学生和村民,尊重当地的民族文化和信仰;
\end{enumerate}
\end{appendix}
\end{document}

