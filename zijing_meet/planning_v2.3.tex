\documentclass[12pt]{ctexart}
% set the left/right margin such that the main title can be written within one line
\usepackage[left=15mm, right=15mm]{geometry}
\usepackage{enumitem}
\AddEnumerateCounter{\chinese}{\chinese}{}
\usepackage{fancyhdr}
\usepackage{longtable}
\usepackage{graphicx}
\usepackage[bottom]{footmisc}
\usepackage{longtable}
\fancypagestyle{runningpage}
{
  \fancyhead{}
  \fancyhead[C]{清华大学深圳研究生院紫荆志愿者团}
  \fancyfoot{}
  \fancyfoot[C]{第 \thepage 页}
}
% not works?
\ctexset {
 appendixname = {附录}
}
\def\CurriculumScheduleWidth{1.6cm}
\begin{document}
% first page is cover
\title{
    \vspace{-0.5in}
    \textmd{\textbf{\huge{2018秋季学期清华大学深圳研究生院紫荆志愿者团\quad 茶话会策划}}}\\
    \normalsize\vspace{0.1in}\Large{2018年秋季学期}\\
    \vspace{1in}
     \textbf{\huge{策}}\\
    \vspace{1in}
     \textbf{\huge{划}}\\
    \vspace{1in}
     \textbf{\huge{书}}\\
    \vspace{1in}
}
\author{策划人:赵丰、黄少平、姜伟峰}
\maketitle
\thispagestyle{empty}
\pagebreak
\pagestyle{runningpage}

\section{项目背景}
\begin{itemize}
\item 团委招新后,清华大学深圳研究生院紫荆志愿团确定了10名正式的志愿团新同学,加上赵丰邀请了几位之前参加活动比较积极的同学,以及三位老团员,还有隔壁哈工大义工联的负责同学,希望通过一个集中的时间大家相互认识一下,并策划新学期联合开展的活动。
\item 志愿团参与实践答辩分工【小组】。
\item 与哈工大义工联负责人交流【两位到得比较早,会前交流以及会后交流,或利用荔园同学回宿舍的契机】。
\end{itemize}
\section{项目目标}
\begin{itemize}
\item 【重要】相互认识。
\item 志愿团参与实践答辩分工。
\item 【重要】提出一个活动想法,与哈工大义工联进一步沟通,给团委老师汇报。
\end{itemize}
\section{项目简介}
\begin{itemize}
\item 主办方:清华大学深圳研究生院紫荆志愿者团
\item 地点:清华大学深圳研究生院Q405
\item 时间:9月12日22:00-23:00
\item 参与人数: 清华大学深圳研究生院学生 16 人、哈工大义工联2人
\begin{longtable}{|c|c|c|p{8cm}|}
\hline
姓名 & 性别 & 年级 & \multicolumn{1}{|c|}{简介\footnote{仅列举紫荆志愿团开展的活动}}\\
\hline
赵丰 & 男 & 研2 & 18年春团委招新入团,参与金色年华、英语演讲比赛志愿、清明纪念丛飞、校庆志愿、半马志愿2次、组织75募捐、暑期爱心支教、迎新志愿、义工证志愿\\
\hline
吴琳 & 女 & 研1  & 18年秋团委招新入团,参与义工证志愿\\
\hline
张奕婷 & 女 & 研1 & 18年秋团委招新入团\\
\hline
王新亮 & 男 & 研1 & 18年秋团委招新入团\\
\hline
黄少平 & 男 & 研2 & 18年夏参与支教入团,参与暑期爱心支教、团建教练 \\
%\hline
%王鲜俐 & 女 & 研2 & 18年夏参与支教入团,参与暑期爱心支教、团建教练 \\
\hline
熊振文【迟到】 & 男 & 研1 & 18年秋参与活动入团,参与迎新志愿、义工证志愿 \\
\hline
姜伟峰 & 男 & 研1 & 18年秋团委招新入团,参与迎新志愿 \\
\hline
孙策 & 男 & 研1 & 18年秋团委招新入团,参与义工证志愿 \\
\hline
张晓冰 & 男 & 研2 & 17年秋团委招新入团,参与义工证志愿2次、清明纪念丛飞、半马志愿、暑期爱心支教、迎新志愿、义工证志愿,组织节水志愿、英语演讲比赛志愿、
校庆志愿、75募捐 \\
\hline
李琳【没来】 & 女 & 研2 & 17年秋团委招新入团,参与义工证志愿(17年)、金色年华2次、英语演讲比赛志愿、校庆志愿、75募捐、迎新志愿 \\
\hline
高艺华 & 女 & 研1 & 18年秋团委招新入团,参与义工证志愿 \\
\hline
郭慧 & 女 & 研1 & 18年秋团委招新入团,参与义工证志愿 \\
\hline
刘舒婷 & 女 & 研1 & 18年秋团委招新入团,参与迎新志愿、义工证志愿,订做义工服 \\
\hline
侯利兵 & 男 & 研1 & 18年秋参与活动入团,参与迎新志愿、义工证志愿 \\
%\hline
%安鑫 & 男 & 研1  & 参与义工证志愿  \\
\hline
胡泰峰 & 男 & 研1 & 18年秋参与活动入团,参与义工证志愿 \\
\hline
张昕阳 【没来】& 女 & 研1 & 18年春团委招新入团,参与英语演讲比赛志愿 \\
\hline
鄂倩倩 【没来】& 女 & 研1 & 18年秋参与活动入团,参与迎新志愿、义工证志愿 \\
\hline
王一璠【有事没来】  & 男 & 研1 & 18年秋团委招新入团,参与迎新志愿、义工证志愿 \\
\hline
何彦平 & 男 & 研1 &  18年秋参与活动入团,参与义工证志愿 \\
\hline
刘柯迪【没来】 & 男 & 研1 & 18年秋参与活动入团,参与迎新志愿、义工证志愿,制作相关推送3篇 \\
\hline
刘崇琦 & 女 & 大三 & 哈工大义工联,参与半马志愿 \\
\hline
涂志华 & 女 & 大三 & 哈工大义工联 \\
\hline
张炎武【迟到很久】 & 男 & 研二 & 大学城志愿者部,参与暑期爱心支教 \\
\hline
\end{longtable}
\end{itemize}
\section{项目准备}
\begin{enumerate}
\item 策划
\item 借教室
\item 实施
\end{enumerate}
\section{项目流程}
\begin{table}[!ht]
\centering
\begin{tabular}{|c|p{8cm}|}
%【赵丰表演手语舞“一起走”】
\hline
时间 & 安排 \\
\hline
22点05  & 自我介绍【包含志愿故事分享】,相互认识 \\
\hline
22点40 & 张晓冰介绍志愿团工作,赵丰补充。\\
%\hline
%22点30 &  头脑风暴【赵丰主持】  \\
\hline
22点50  & 张炎武介绍大学城志愿者部 \\
\hline
22点55 & 集体合影 \\
\hline
23点 & 张晓冰、张炎武、黄少平、赵丰留下、支教实践队员白杨、王鲜俐参与讨论整材料分工;其他同学散会。   \\
\hline
\end{tabular}
\caption{时间安排表}\label{route}
\end{table}



\section{项目后续阶段}
\begin{enumerate}
\item 15日拾行实施
\item 支教答辩确定9月17号预答辩人选
%\item 探讨住荔园的同学是否可以参加工大义工联活动的事情
%\item 头脑风暴一个idea 成型后上报团委老师
\end{enumerate}
\section{联系方式}
清华大学深圳研究生院紫荆志愿者团副团长:赵丰

Tel: 18800190762

Email: 616545598@qq.com
\begin{flushright}
清华大学深圳研究生院紫荆志愿者团\\
\the\year 年 \the\month 月 \the\day 日
\end{flushright}
\begin{appendix}
\section{关于院紫荆志愿者团}
清华大学深圳研究生院紫荆志愿者服务团是院团委直属的学生志愿服务组织,以“自我实践、服务他人、自我教育、推动社会”为宗旨,旨在组织、协调我院研究生志愿者活动,开展具有清华特色的丰富多彩的研究生志愿活动。
\begin{enumerate}[label = {\chinese*、}]
\item 常规志愿活动
\begin{enumerate}[label =(\arabic*)]
         \item  清华大学深圳研究生院新生集中办理义工证活动
         \item 与爱同行·关爱“星星的后裔“志愿活动
         \item  深圳高校“河未来、益起行”节水探源活动
         \item 院五星级志愿者评选活动
\end{enumerate}
\item 特色志愿活动
\begin{enumerate}[label =(\arabic*)]
        \item 大型赛会志愿服务活动
        \item 学院重点活动(如迎新、校庆、大学城半程马拉松、教工迎新年晚会等)志愿服务
        \item 面向留学生的志愿讲解员培养
\end{enumerate}
\item 突破性活动
\begin{enumerate}[label =(\arabic*)]
         \item 暑期爱心支教
\end{enumerate}
\end{enumerate}
%\section{一起走歌词}
%\begin{center}
%伸出彼此的双手 \\
%传递心中的热流 \\
%因为爱我们来到一起 \\
%身边到处都是朋友 \\
%\quad\\
%世上有许多感动 \\
%人间有许多温柔 \\
%当风雨终于变成彩虹 \\
%付出就是一种拥有 \\
%\quad\\
%一起走 \\
%不需要理由\\
%一起走\\
%把美丽守候\\
%尽我所能\\
%无取无求\\
%笑脸如花\\
%盛开到永久\\
%\end{center}
\section{项目物资}\label{scheduling}
\begin{enumerate}
\item 零食和饮料【赵丰从团委申请到¥150】
\end{enumerate}
%\section{关于头脑风暴}
%\begin{itemize}
%\item 主持人:赵丰
%\item 简介:集中大家的智慧,提出一个可实施的\textbf{大学城校园公益}项目。
%\item 活动规则:首先分小组讨论,每个小组成员可以提前准备好一个idea, 组内讨论出一个可以利用校内外资源实现的 idea 。展示中要涉及项目愿景和实施的大概方案。
%其他同学可就项目提问。投票环节按个人投票,每人一票,不能投自己小组。得票最高的项目细化后汇报给团委老师推动实施。
%\item 活动流程: 
%\begin{enumerate}
%\item 10min 分组讨论
%\item 3min 展示 + 1.5min 提问 
%\item 2min 投票
%\end{enumerate}
%\item 活动物资:参与者提前准备纸笔,主办方准备Mark笔和移动白板。
%\item 随机分成4组%按宿舍楼分组,再调整

%\begin{table}[!ht]
%\centering
%\begin{tabular}{|c|c|}
%\hline
%【荷1】& 张晓冰、侯利兵、孙策、胡泰峰、郭慧(荷5)\\
%\hline
%【荷2】& 李琳、王鲜俐、熊振文、高艺华、姜伟峰、刘舒婷、刘柯迪、黄少平、张莉\\
%\hline
%【荔园】& 王新亮、王一璠、吴琳、张奕婷、何彦平、张昕阳、刘崇琦、涂志华、安鑫\\
%\hline
%\end{tabular}
%\caption{分组表格}
%\end{table}
%\end{itemize}
\section{紫荆志愿团现阶段工作介绍}
会上团长张晓冰首先对大家热心志愿服务表示感谢,然后回顾了自己一年多来在团里的工作情况,介绍了紫荆志愿团现阶段的主要工作,并对之后开展的工作传承性问题发表了看法。

副团长赵丰补充了几点:紫荆志愿团现有的定位是立足校内,服务师生。
主要办两类活动,一是招募同学参加院其他部门安排的志愿工作,服务学校;二是组织一些校内志愿活动,服务同学。

回顾半年来的活动,我们有很多成就,例如我们举办了第一届英语演讲比赛,初步培养了深研院自己的校园讲解队伍;
我们第一次在院团委自主立项了暑期支教,让紫荆志愿的旗帜飘扬到了教育扶贫需要我们的地方。

我们也有很多不足,正如团长张晓冰指出的,清华深研院的同学课业科研压力大,参与志愿活动的积极性很难调动;二是缺少特色的校内志愿活动。

哈工大义工联和紫荆志愿团只隔了一条河,是哈工大深圳校团委直属的志愿服务组织,在上学期半程马拉松赛事服务中我们有了初步的接触。希望之后能密切合作,一起做大学城公益。
\section{志愿者保障和守则}
\subsection{保障}
\subsection{守则}
\begin{enumerate}[label = {(\chinese*)}]
\item 具有完全民事行为能力的深研院研究生;
\item 认同紫荆志愿团理念和文化;
\item 有社会责任感和奉献精神,无不良嗜好;
\item 有较强的团结协作能力、语言表达能力和人际交往能力,尊重队友;
\item 有较强的心理素质,在艰苦的环境下能自我调节,保持积极乐观心态,肯吃苦,能坚持,不给项目及队友带来负面影响; 
\item 有爱心,热心帮助他人,尊重他人的民族文化和信仰;
\end{enumerate}
\end{appendix}
\end{document}

